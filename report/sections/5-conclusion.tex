In this project, we started with the classic 2D Tetris using tetrominoes (four-block pieces) and successfully trained an agent to complete the environment using Deep Q-Networks (DQN). We then increased the difficulty by introducing pentominoes (five-block pieces). Although we adjusted various training parameters to help the agent adapt, the overall performance was not as strong—likely due to certain pentomino shapes being inherently more difficult to work with.

Building upon the 2D foundation, we extended the environment into 3D, creating a three-dimensional Tetris game. We selected four types of 3D tetrominoes for training with DQN. The results were impressive: the trained agent achieved average scores exceeding 1,000 points, significantly outperforming random strategies. This also marks one of the first successful applications of DQN to a 3D Tetris environment, demonstrating strong generalization and potential for future development.