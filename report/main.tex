\documentclass{article}

% if you need to pass options to natbib, use, e.g.:
%     \PassOptionsToPackage{numbers, compress}{natbib}
% before loading neurips_2025

% The authors should use one of these tracks.
% Before accepting by the NeurIPS conference, select one of the options below.
% 0. "default" for submission
 \usepackage{neurips_2025}
% the "default" option is equal to the "main" option, which is used for the Main Track with double-blind reviewing.
% 1. "main" option is used for the Main Track
%  \usepackage[main]{neurips_2025}
% 2. "position" option is used for the Position Paper Track
%  \usepackage[position]{neurips_2025}
% 3. "dandb" option is used for the Datasets & Benchmarks Track
 % \usepackage[dandb]{neurips_2025}
% 4. "creativeai" option is used for the Creative AI Track
%  \usepackage[creativeai]{neurips_2025}
% 5. "sglblindworkshop" option is used for the Workshop with single-blind reviewing
 % \usepackage[sglblindworkshop]{neurips_2025}
% 6. "dblblindworkshop" option is used for the Workshop with double-blind reviewing
%  \usepackage[dblblindworkshop]{neurips_2025}

% After being accepted, the authors should add "final" behind the track to compile a camera-ready version.
% 1. Main Track
 % \usepackage[main, final]{neurips_2025}
% 2. Position Paper Track
%  \usepackage[position, final]{neurips_2025}
% 3. Datasets & Benchmarks Track
 % \usepackage[dandb, final]{neurips_2025}
% 4. Creative AI Track
%  \usepackage[creativeai, final]{neurips_2025}
% 5. Workshop with single-blind reviewing
%  \usepackage[sglblindworkshop, final]{neurips_2025}
% 6. Workshop with double-blind reviewing
%  \usepackage[dblblindworkshop, final]{neurips_2025}
% Note. For the workshop paper template, both \title{} and \workshoptitle{} are required, with the former indicating the paper title shown in the title and the latter indicating the workshop title displayed in the footnote.
% For workshops (5., 6.), the authors should add the name of the workshop, "\workshoptitle" command is used to set the workshop title.
% \workshoptitle{WORKSHOP TITLE}

% "preprint" option is used for arXiv or other preprint submissions
 % \usepackage[preprint]{neurips_2025}

% to avoid loading the natbib package, add option nonatbib:
%    \usepackage[nonatbib]{neurips_2025}

\usepackage[utf8]{inputenc} % allow utf-8 input
\usepackage[T1]{fontenc}    % use 8-bit T1 fonts
\usepackage{hyperref}       % hyperlinks
\usepackage{url}            % simple URL typesetting
\usepackage{booktabs}       % professional-quality tables
\usepackage{amsfonts}       % blackboard math symbols
\usepackage{nicefrac}       % compact symbols for 1/2, etc.
\usepackage{microtype}      % microtypography
\usepackage{xcolor}         % colors

% Note. For the workshop paper template, both \title{} and \workshoptitle{} are required, with the former indicating the paper title shown in the title and the latter indicating the workshop title displayed in the footnote. 
\title{Formatting Instructions For NeurIPS 2025}


% The \author macro works with any number of authors. There are two commands
% used to separate the names and addresses of multiple authors: \And and \AND.
%
% Using \And between authors leaves it to LaTeX to determine where to break the
% lines. Using \AND forces a line break at that point. So, if LaTeX puts 3 of 4
% authors names on the first line, and the last on the second line, try using
% \AND instead of \And before the third author name.


\author{%
  David S.~Hippocampus\thanks{Use footnote for providing further information
    about author (webpage, alternative address)---\emph{not} for acknowledging
    funding agencies.} \\
  Department of Computer Science\\
  Cranberry-Lemon University\\
  Pittsburgh, PA 15213 \\
  \texttt{hippo@cs.cranberry-lemon.edu} \\
  % examples of more authors
  % \And
  % Coauthor \\
  % Affiliation \\
  % Address \\
  % \texttt{email} \\
  % \AND
  % Coauthor \\
  % Affiliation \\
  % Address \\
  % \texttt{email} \\
  % \And
  % Coauthor \\
  % Affiliation \\
  % Address \\
  % \texttt{email} \\
  % \And
  % Coauthor \\
  % Affiliation \\
  % Address \\
  % \texttt{email} \\
}


\begin{document}

\maketitle

\begin{abstract}
    The abstract paragraph should be indented \nicefrac{1}{2}~inch (3~picas) on
    both the left- and right-hand margins. Use 10~point type, with a vertical
    spacing (leading) of 11~points.  The word \textbf{Abstract} must be centered,
    bold, and in point size 12. Two line spaces precede the abstract. The abstract
    must be limited to one paragraph.
  \end{abstract}

\section{Introduction}
Recent research on game strategy agents has flourished in response to the growing demand for intelligent systems capable 
of playing strategic games either alongside or against human players. 
Reinforcement learning (RL) has established itself as a powerful paradigm for training such agents, enabling them to acquire 
optimal behaviors through interaction with an environment to maximize cumulative rewards over time.
A variety of algorithmic advancements have been proposed for a comprehensive generalist mastering diverse games within a unified framework. 
\textbf{Deep Q-Network (DQN)} \cite{mnih2013playingatarideepreinforcement} pioneered the use of deep neural networks to approximate value 
functions, enabling end-to-end learning from raw pixel inputs. 
\textbf{Proximal Policy Optimization (PPO)} \cite{schulman2017proximalpolicyoptimizationalgorithms} introduced a more stable and 
sample-efficient policy gradient method, supporting large-scale training through self-play and mini-batch updates from stored trajectories.

Building on these foundations, \textbf{AlphaZero} \cite{silver2017masteringchessshogiselfplay} demonstrated the power of combining deep 
learning with tree-based planning in a model-based fashion, but it relied on known environment dynamics—most notably, the rules of the game. 
\textbf{MuZero} \cite{Schrittwieser2020} advanced this line of work by learning not only the policy and value functions but also a latent, 
implicit model of the environment’s dynamics. 
This abstraction allowed \textbf{MuZero} to extend the planning-based advantages of \textbf{AlphaZero} to previously unknown or complex 
domains, where explicit modeling is impractical. 
MuZero represents a paradigm shift in model-based RL by learning not only the value and policy functions, but also the dynamics of the 
environment without access to the actual game rules. Its success has been demonstrated across a diverse range of domains, including board 
games like Go, Chess, and Shogi, as well as in more complex and dynamic environments such as Atari. 
MuZero’s ability to integrate planning with learned models and to perform well in both deterministic and stochastic environments positions 
it as one of the most general and powerful agents to date.

As part of our investigation into MuZero's adaptability to varied strategic domains, we selected \emph{Connect6}, a relatively underexplored game 
known for its balanced mechanics and heightened complexity relative to classic games like \emph{Gomoku}. 
\emph{Connect6} addresses the first-player advantage problem inherent in many connection games by introducing the continuous dual-step rule. 
This subtle change significantly increases the strategic depth and makes the game an ideal candidate for advanced AI research. 
Though prior works leverage \textbf{Monte Carlo Tree Search} \cite{5740585} or \textbf{AlphaZero} \cite{Yang2020}, most research relies on hand-crafted 
heuristics and incorporates explicit rule features to structure the training framework.
In this context, we conduct the experiments with respect to application of \textbf{MuZero} to \emph{Connect6}, aiming to evaluate its performance and 
adaptability in a novel and partially observable environment. 
This research sheds new light on \textbf{MuZero}’s scalability and progressively helps advance in game-playing intelligent agents.

\input{sections/2-related_work}

\input{sections/3-Methodology}

\input{sections/4-results}

In this project, we started with the classic 2D Tetris using tetrominoes (four-block pieces) and successfully trained an agent to complete the environment using Deep Q-Networks (DQN). We then increased the difficulty by introducing pentominoes (five-block pieces). Although we adjusted various training parameters to help the agent adapt, the overall performance was not as strong—likely due to certain pentomino shapes being inherently more difficult to work with.

Building upon the 2D foundation, we extended the environment into 3D, creating a three-dimensional Tetris game. We selected four types of 3D tetrominoes for training with DQN. The results were impressive: the trained agent achieved average scores exceeding 1,000 points, significantly outperforming random strategies. This also marks one of the first successful applications of DQN to a 3D Tetris environment, demonstrating strong generalization and potential for future development.

\bibliographystyle{plain}
\bibliography{custom}

\appendix

\section{Technical Appendices and Supplementary Material}
Technical appendices with additional results, figures, graphs and proofs may be submitted with the paper submission before the full submission deadline (see above), or as a separate PDF in the ZIP file below before the supplementary material deadline. There is no page limit for the technical appendices.


\end{document}